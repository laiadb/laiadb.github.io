%% template.tex
%% from
%% bare_conf.tex
%% V1.4b
%% 2015/08/26
%% by Michael Shell
%% See:
%% http://www.michaelshell.org/
%% for current contact information.
%%
%% This is a skeleton file demonstrating the use of IEEEtran.cls
%% (requires IEEEtran.cls version 1.8b or later) with an IEEE
%% conference paper.
%%
%% Support sites:
%% http://www.michaelshell.org/tex/ieeetran/
%% http://www.ctan.org/pkg/ieeetran
%% and
%% http://www.ieee.org/

%%*************************************************************************
%% Legal Notice:
%% This code is offered as-is without any warranty either expressed or
%% implied; without even the implied warranty of MERCHANTABILITY or
%% FITNESS FOR A PARTICULAR PURPOSE!
%% User assumes all risk.
%% In no event shall the IEEE or any contributor to this code be liable for
%% any damages or losses, including, but not limited to, incidental,
%% consequential, or any other damages, resulting from the use or misuse
%% of any information contained here.
%%
%% All comments are the opinions of their respective authors and are not
%% necessarily endorsed by the IEEE.
%%
%% This work is distributed under the LaTeX Project Public License (LPPL)
%% ( http://www.latex-project.org/ ) version 1.3, and may be freely used,
%% distributed and modified. A copy of the LPPL, version 1.3, is included
%% in the base LaTeX documentation of all distributions of LaTeX released
%% 2003/12/01 or later.
%% Retain all contribution notices and credits.
%% ** Modified files should be clearly indicated as such, including  **
%% ** renaming them and changing author support contact information. **
%%*************************************************************************


% *** Authors should verify (and, if needed, correct) their LaTeX system  ***
% *** with the testflow diagnostic prior to trusting their LaTeX platform ***
% *** with production work. The IEEE's font choices and paper sizes can   ***
% *** trigger bugs that do not appear when using other class files.       ***                          ***
% The testflow support page is at:
% http://www.michaelshell.org/tex/testflow/

\documentclass[conference,final,]{IEEEtran}
% Some Computer Society conferences also require the compsoc mode option,
% but others use the standard conference format.
%
% If IEEEtran.cls has not been installed into the LaTeX system files,
% manually specify the path to it like:
% \documentclass[conference]{../sty/IEEEtran}





% Some very useful LaTeX packages include:
% (uncomment the ones you want to load)


% *** MISC UTILITY PACKAGES ***
%
%\usepackage{ifpdf}
% Heiko Oberdiek's ifpdf.sty is very useful if you need conditional
% compilation based on whether the output is pdf or dvi.
% usage:
% \ifpdf
%   % pdf code
% \else
%   % dvi code
% \fi
% The latest version of ifpdf.sty can be obtained from:
% http://www.ctan.org/pkg/ifpdf
% Also, note that IEEEtran.cls V1.7 and later provides a builtin
% \ifCLASSINFOpdf conditional that works the same way.
% When switching from latex to pdflatex and vice-versa, the compiler may
% have to be run twice to clear warning/error messages.






% *** CITATION PACKAGES ***
%
%\usepackage{cite}
% cite.sty was written by Donald Arseneau
% V1.6 and later of IEEEtran pre-defines the format of the cite.sty package
% \cite{} output to follow that of the IEEE. Loading the cite package will
% result in citation numbers being automatically sorted and properly
% "compressed/ranged". e.g., [1], [9], [2], [7], [5], [6] without using
% cite.sty will become [1], [2], [5]--[7], [9] using cite.sty. cite.sty's
% \cite will automatically add leading space, if needed. Use cite.sty's
% noadjust option (cite.sty V3.8 and later) if you want to turn this off
% such as if a citation ever needs to be enclosed in parenthesis.
% cite.sty is already installed on most LaTeX systems. Be sure and use
% version 5.0 (2009-03-20) and later if using hyperref.sty.
% The latest version can be obtained at:
% http://www.ctan.org/pkg/cite
% The documentation is contained in the cite.sty file itself.






% *** GRAPHICS RELATED PACKAGES ***
%
\ifCLASSINFOpdf
  % \usepackage[pdftex]{graphicx}
  % declare the path(s) where your graphic files are
  % \graphicspath{{../pdf/}{../jpeg/}}
  % and their extensions so you won't have to specify these with
  % every instance of \includegraphics
  % \DeclareGraphicsExtensions{.pdf,.jpeg,.png}
\else
  % or other class option (dvipsone, dvipdf, if not using dvips). graphicx
  % will default to the driver specified in the system graphics.cfg if no
  % driver is specified.
  % \usepackage[dvips]{graphicx}
  % declare the path(s) where your graphic files are
  % \graphicspath{{../eps/}}
  % and their extensions so you won't have to specify these with
  % every instance of \includegraphics
  % \DeclareGraphicsExtensions{.eps}
\fi
% graphicx was written by David Carlisle and Sebastian Rahtz. It is
% required if you want graphics, photos, etc. graphicx.sty is already
% installed on most LaTeX systems. The latest version and documentation
% can be obtained at:
% http://www.ctan.org/pkg/graphicx
% Another good source of documentation is "Using Imported Graphics in
% LaTeX2e" by Keith Reckdahl which can be found at:
% http://www.ctan.org/pkg/epslatex
%
% latex, and pdflatex in dvi mode, support graphics in encapsulated
% postscript (.eps) format. pdflatex in pdf mode supports graphics
% in .pdf, .jpeg, .png and .mps (metapost) formats. Users should ensure
% that all non-photo figures use a vector format (.eps, .pdf, .mps) and
% not a bitmapped formats (.jpeg, .png). The IEEE frowns on bitmapped formats
% which can result in "jaggedy"/blurry rendering of lines and letters as
% well as large increases in file sizes.
%
% You can find documentation about the pdfTeX application at:
% http://www.tug.org/applications/pdftex





% *** MATH PACKAGES ***
%
%\usepackage{amsmath}
% A popular package from the American Mathematical Society that provides
% many useful and powerful commands for dealing with mathematics.
%
% Note that the amsmath package sets \interdisplaylinepenalty to 10000
% thus preventing page breaks from occurring within multiline equations. Use:
%\interdisplaylinepenalty=2500
% after loading amsmath to restore such page breaks as IEEEtran.cls normally
% does. amsmath.sty is already installed on most LaTeX systems. The latest
% version and documentation can be obtained at:
% http://www.ctan.org/pkg/amsmath





% *** SPECIALIZED LIST PACKAGES ***
%
%\usepackage{algorithmic}
% algorithmic.sty was written by Peter Williams and Rogerio Brito.
% This package provides an algorithmic environment fo describing algorithms.
% You can use the algorithmic environment in-text or within a figure
% environment to provide for a floating algorithm. Do NOT use the algorithm
% floating environment provided by algorithm.sty (by the same authors) or
% algorithm2e.sty (by Christophe Fiorio) as the IEEE does not use dedicated
% algorithm float types and packages that provide these will not provide
% correct IEEE style captions. The latest version and documentation of
% algorithmic.sty can be obtained at:
% http://www.ctan.org/pkg/algorithms
% Also of interest may be the (relatively newer and more customizable)
% algorithmicx.sty package by Szasz Janos:
% http://www.ctan.org/pkg/algorithmicx




% *** ALIGNMENT PACKAGES ***
%
%\usepackage{array}
% Frank Mittelbach's and David Carlisle's array.sty patches and improves
% the standard LaTeX2e array and tabular environments to provide better
% appearance and additional user controls. As the default LaTeX2e table
% generation code is lacking to the point of almost being broken with
% respect to the quality of the end results, all users are strongly
% advised to use an enhanced (at the very least that provided by array.sty)
% set of table tools. array.sty is already installed on most systems. The
% latest version and documentation can be obtained at:
% http://www.ctan.org/pkg/array


% IEEEtran contains the IEEEeqnarray family of commands that can be used to
% generate multiline equations as well as matrices, tables, etc., of high
% quality.




% *** SUBFIGURE PACKAGES ***
%\ifCLASSOPTIONcompsoc
%  \usepackage[caption=false,font=normalsize,labelfont=sf,textfont=sf]{subfig}
%\else
%  \usepackage[caption=false,font=footnotesize]{subfig}
%\fi
% subfig.sty, written by Steven Douglas Cochran, is the modern replacement
% for subfigure.sty, the latter of which is no longer maintained and is
% incompatible with some LaTeX packages including fixltx2e. However,
% subfig.sty requires and automatically loads Axel Sommerfeldt's caption.sty
% which will override IEEEtran.cls' handling of captions and this will result
% in non-IEEE style figure/table captions. To prevent this problem, be sure
% and invoke subfig.sty's "caption=false" package option (available since
% subfig.sty version 1.3, 2005/06/28) as this is will preserve IEEEtran.cls
% handling of captions.
% Note that the Computer Society format requires a larger sans serif font
% than the serif footnote size font used in traditional IEEE formatting
% and thus the need to invoke different subfig.sty package options depending
% on whether compsoc mode has been enabled.
%
% The latest version and documentation of subfig.sty can be obtained at:
% http://www.ctan.org/pkg/subfig




% *** FLOAT PACKAGES ***
%

%\usepackage{fixltx2e}
% fixltx2e, the successor to the earlier fix2col.sty, was written by
% Frank Mittelbach and David Carlisle. This package corrects a few problems
% in the LaTeX2e kernel, the most notable of which is that in current
% LaTeX2e releases, the ordering of single and double column floats is not
% guaranteed to be preserved. Thus, an unpatched LaTeX2e can allow a
% single column figure to be placed prior to an earlier double column
% figure.
% Be aware that LaTeX2e kernels dated 2015 and later have fixltx2e.sty's
% corrections already built into the system in which case a warning will
% be issued if an attempt is made to load fixltx2e.sty as it is no longer
% needed.
% The latest version and documentation can be found at:
% http://www.ctan.org/pkg/fixltx2e


%\usepackage{stfloats}
% stfloats.sty was written by Sigitas Tolusis. This package gives LaTeX2e
% the ability to do double column floats at the bottom of the page as well
% as the top. (e.g., "\begin{figure*}[!b]" is not normally possible in
% LaTeX2e). It also provides a command:
%\fnbelowfloat
% to enable the placement of footnotes below bottom floats (the standard
% LaTeX2e kernel puts them above bottom floats). This is an invasive package
% which rewrites many portions of the LaTeX2e float routines. It may not work
% with other packages that modify the LaTeX2e float routines. The latest
% version and documentation can be obtained at:
% http://www.ctan.org/pkg/stfloats
% Do not use the stfloats baselinefloat ability as the IEEE does not allow
% \baselineskip to stretch. Authors submitting work to the IEEE should note
% that the IEEE rarely uses double column equations and that authors should try
% to avoid such use. Do not be tempted to use the cuted.sty or midfloat.sty
% packages (also by Sigitas Tolusis) as the IEEE does not format its papers in
% such ways.
% Do not attempt to use stfloats with fixltx2e as they are incompatible.
% Instead, use Morten Hogholm'a dblfloatfix which combines the features
% of both fixltx2e and stfloats:
%
% \usepackage{dblfloatfix}
% The latest version can be found at:
% http://www.ctan.org/pkg/dblfloatfix




% *** PDF, URL AND HYPERLINK PACKAGES ***
%
%\usepackage{url}
% url.sty was written by Donald Arseneau. It provides better support for
% handling and breaking URLs. url.sty is already installed on most LaTeX
% systems. The latest version and documentation can be obtained at:
% http://www.ctan.org/pkg/url
% Basically, \url{my_url_here}.




% *** Do not adjust lengths that control margins, column widths, etc. ***
% *** Do not use packages that alter fonts (such as pslatex).         ***
% There should be no need to do such things with IEEEtran.cls V1.6 and later.
% (Unless specifically asked to do so by the journal or conference you plan
% to submit to, of course. )



%% BEGIN MY ADDITIONS %%


\usepackage{graphicx}
% We will generate all images so they have a width \maxwidth. This means
% that they will get their normal width if they fit onto the page, but
% are scaled down if they would overflow the margins.
\makeatletter
\def\maxwidth{\ifdim\Gin@nat@width>\linewidth\linewidth
\else\Gin@nat@width\fi}
\makeatother
\let\Oldincludegraphics\includegraphics
\renewcommand{\includegraphics}[1]{\Oldincludegraphics[width=\maxwidth]{#1}}

\usepackage[unicode=true]{hyperref}

\hypersetup{
            pdftitle={The LAIA Database -- A Secure and Decentralized Platform for Environmental Data},
            pdfborder={0 0 0},
            breaklinks=true}
\urlstyle{same}  % don't use monospace font for urls

% Pandoc toggle for numbering sections (defaults to be off)
\setcounter{secnumdepth}{0}

% Pandoc syntax highlighting

% Pandoc header

\providecommand{\tightlist}{%
  \setlength{\itemsep}{0pt}\setlength{\parskip}{0pt}}

%% END MY ADDITIONS %%


\hyphenation{op-tical net-works semi-conduc-tor}

\begin{document}
%
% paper title
% Titles are generally capitalized except for words such as a, an, and, as,
% at, but, by, for, in, nor, of, on, or, the, to and up, which are usually
% not capitalized unless they are the first or last word of the title.
% Linebreaks \\ can be used within to get better formatting as desired.
% Do not put math or special symbols in the title.
\title{The LAIA Database -- A Secure and Decentralized Platform for
Environmental Data}

% author names and affiliations
% use a multiple column layout for up to three different
% affiliations

\author{

%% ---- classic IEEETrans wide authors' list ----------------
 % -- end affiliation.wide
%% ----------------------------------------------------------



%% ---- classic IEEETrans one column per institution --------
 %% -- beg if/affiliation.institution-columnar
\IEEEauthorblockN{
 %% -- end for/affiliation.institution.author
}
\IEEEauthorblockA{Center for Social Ecology and Free Technology\\
SEFT
\\seft@riseup.net
}
 %% -- end for/affiliation.institution
 %% -- end if/affiliation.institution-columnar
%% ----------------------------------------------------------





%% ---- one column per author, classic/default IEEETrans ----
 %% -- end if/affiliation.institution-columnar
%% ----------------------------------------------------------

}

% conference papers do not typically use \thanks and this command
% is locked out in conference mode. If really needed, such as for
% the acknowledgment of grants, issue a \IEEEoverridecommandlockouts
% after \documentclass

% for over three affiliations, or if they all won't fit within the width
% of the page, use this alternative format:
%
%\author{\IEEEauthorblockN{Michael Shell\IEEEauthorrefmark{1},
%Homer Simpson\IEEEauthorrefmark{2},
%James Kirk\IEEEauthorrefmark{3},
%Montgomery Scott\IEEEauthorrefmark{3} and
%Eldon Tyrell\IEEEauthorrefmark{4}}
%\IEEEauthorblockA{\IEEEauthorrefmark{1}School of Electrical and Computer Engineering\\
%Georgia Institute of Technology,
%Atlanta, Georgia 30332--0250\\ Email: see http://www.michaelshell.org/contact.html}
%\IEEEauthorblockA{\IEEEauthorrefmark{2}Twentieth Century Fox, Springfield, USA\\
%Email: homer@thesimpsons.com}
%\IEEEauthorblockA{\IEEEauthorrefmark{3}Starfleet Academy, San Francisco, California 96678-2391\\
%Telephone: (800) 555--1212, Fax: (888) 555--1212}
%\IEEEauthorblockA{\IEEEauthorrefmark{4}Tyrell Inc., 123 Replicant Street, Los Angeles, California 90210--4321}}




% use for special paper notices
%\IEEEspecialpapernotice{(Invited Paper)}




% make the title area
\maketitle

% As a general rule, do not put math, special symbols or citations
% in the abstract
\begin{abstract}
The current centralized storage of data in private servers exposes the
whole society to the threats of surveillance, censorship and malicious
attacks coming from different sources. Under this context, the environmental
and climate data are also a target of these threats, therefore a shift on
the topology of data storage and sharing is needed. The present article
proposes the LAIA Database, a decentralized and encrypted platform focused
on environmental and climate data that provides a total control over the data
by the users. LAIA is a friend-to-friend network (F2F), which only allows access
to the data by trusted nodes on the network. The project aims to improve the
research collaboration between groups as it allows to share important
information between trusted parts, while it protects both research
groups and their data against censorship, persecution and other forms of
political retaliation.
\end{abstract}

% no keywords

% use for special paper notices



% make the title area
\maketitle

% no keywords

% For peer review papers, you can put extra information on the cover
% page as needed:
% \ifCLASSOPTIONpeerreview
% \begin{center} \bfseries EDICS Category: 3-BBND \end{center}
% \fi
%
% For peerreview papers, this IEEEtran command inserts a page break and
% creates the second title. It will be ignored for other modes.
\IEEEpeerreviewmaketitle


\hypertarget{introduction}{%
\section{Introduction}\label{introduction}}

The concepts of privacy, anonymity and digital self-defense are not only
restricted to activists and groups directly involved with digital rights
and other political movements. Since the surveillance apparatuses of the
state were revealed in detail by Edward Snowden, many social movements
started to deal with technological devices in a different way. Back
then, the massive storage of data in a centralized database under US
jurisdiction came out to the public spot. The issue of mass surveillance
not only affected citizens but also other nation-states depending on
USA-based infrastructures, which eventually turned out into an
international crisis (Sargsyan 2016).

Recently there has been a rise of critical voices against investments in
environmental and climate research programs. For some of them,
environmental policies are seen as a threat to industries and the
national economy, this is something on which many politician
base their rhetoric (Trump, for instance, as a climate-change-denier
based his campaign on this issue with the slogan ``make America great
again''). Once in power conservative politicians supported by
climate-change-deniers have moved from a mere rhetoric into
political actions across the globe (De Pryck and Gemenne 2017). This new
political scenario resulted in a strong reaction from researchers and
activists that started to take some measures to protect the
data from being altered, deleted or removed (an example of these
measures was the mass-download of US government climate data before
Donald Trump's administration took over). Nonetheless, what it is seen
is a shift of data storage from US-based servers to servers under
different jurisdiction, but owned by private groups and big corporations
keeping high centralization of the knowledge on the Web in a few large
providers (Halpin and Monnin 2016). Hence, although this action intends
to rescue the data, its strategy is very ineffective as the political
scenario is constantly changing in accordance with centralized interests.

A more efficient and low cost way to protect the data is based on a real
paradigmatic shift on data storage, from a centralized and vulnerable
structure to a decentralized and serverless architecture. This is a path
to recover the autonomy and control over environmental data by reducing
the risks of data loss, the single point of failure and the exploitation
of centralized data controlled by corporations and governments. This
distributed structure is also social, where the autonomy is associated
with the strength of the social bonds and the flux of information. Even 
though there are already some decentralized projects available (e.g
BitTorrent, git, etc), none of them was developed focusing on climate
and environmental data.

The current improvement of environmental data generation, from universities
and research institutes to citizen scientific labs and collectives, contrasts
with the lack of an integrated platform infrastructure for data publishing
and the communication between different groups involved with the environmental
monitoring is enhanced. Concerning to the limitations and vulnerabilities
of the mainstream database structures, this project aims to develop an
accessible and decentralized database based on trustworthy networks, privacy
and anonymity. Under this structure a central service is not required and
if a node is under attack, the other parts of the network can work without
interruption.

Serverless means that each node is server\&client at the same time,
since no external service is needed for the network to work. Thus,
it's based on the principle of a Network of Participation, which means
that the data will never disappear as long as there is a network of
nodes interested on keep it available in their computer. Moreover, with
the multi-source sharing (swarming) and the multi-hop a small number of
hops between nodes is expected to access files resulting in faster
downloads (low latency).

\hypertarget{the-database-architecture}{%
\section{The Database Architecture}\label{the-database-architecture}}

The LAIA Database is a cross-platform database focused on the user
experience in order to build up a Web of Trust between peers. It
integrates a strong encrypted protocol with usability and gives a total
control over the data by users.
In the decentralized structure of LAIA database the data is hosted in
swarms, where a user who join the swarm become a peer and can download
pieces of the data from other node since establish connection to at
least one trusted peers in the swarm. In return, the user uploads
pieces of data they already have other peers that need it. Once the user
has all of the data, they can choose to stay in the swarm and
continue sharing with other trusted peers, which makes them a seed.
Accordingly, in this network of participation the structure the more
popular the data the bigger the swarm and the faster the downloads.

LAIA database a private peer-to-peer (P2P) in which each network node exchanges
data only with a designated list of ``friend'' nodes. Also known as
Friend-to-Friend (F2F), this network topology builds its overlay on top of
pre-existent trust relationships among its users. A digital life
is a dimension of real-life and this is a very important base of LAIA's
principles, where only trusted nodes establish connection among them
resulting in a more secure network. With F2F it is possible to ensure
a cryptographically secure connection established between nodes that
had met each other in a past experience outside the platform. Thus,
unlike P2P network, with F2F it is possible to control who accesses
each others data through encrypted connections between trusted parts
that had already shared secrets needed to establish these secure connections.

Related to the search for files, it is possible for nodes to look for
files accessing a large network even when still only connected to their
trusted nodes. The trusted node's search works by propagating search and
file transfers recursively to all friends of friends to a certain degree
of separation. This topology ends up with a non-homogeneous structure with
regard to the amount of nodes and bandwidth (Fig. 1), where the
swarming structure ensure either technical or legal means to overcome
traditional ways for limiting the access to the Web (Fig. 2).

The LAIA database uses the reputation scores to give a general overview
of the peers' level of participation in the network. The role of this
tool is to protect nodes against the problems of P2P network with spams
and malicious nodes, preventing disturbing data spreading without
control. Reputation score is accounted for each node considering the
negative score as bad, the positive as good and the zero as neutral. If
the score is too low, the identity is flagged as bad. During the
bootstrapping process for joining the P2P network, peers can potentially
use reputations to decide who to directly connect to in the overlay
topology.

\begin{figure}[htbp]
\centering
\includegraphics{network.png}
\caption{Friend-to-Friend: The network of no-homogeneous networks. As
the swarming is built up based on trust between nodes, the networks
might vary in matter of bandwidth and size}
\end{figure}


\hypertarget{security-and-privacy}{%
\subsection{Security and Privacy}\label{security-and-privacy}}

The profile is created by the generation of a PGP key, where nodes
use a passphrase to validate the profile. At the login, the PGP key
is used to decrypt the encrypted SSL certificate and its passphrase is
read. The SSL passphrase is chosen randomly when creating the location and
encrypted using the users' PGP key.

The data is identified by its specific fingerprint, and unlike other
decentralized projects (e.g.~IPFS) it is also possible to look for files
by their cryptographic hashes, names or even by their sizes. The
security relies on cryptographic algorithms in which the connection
between nodes are encrypted using SSL, while the identity of a node is
represented by its PGP key. Once the software is installed and started
for the first time, a profile and a certificate are automatically
generated. That certificate, in turn, is used to authenticate and
connect safely with the trusted node to include in the network.

Once a folder is added to be shared, all the files present inside it
will be hashed (i.e.~there will be created a fingerprint for each file)
before being shared by the node. This folder can be shared as
``browsable'', in which only trusted nodes can see and download the
shared files, ``network wide'', where friends of friend can download the
files via anonymous tunnels. In this case, the friends of friends nodes
are not capable to see which files are being shared and the node that is
the source of these files, but they can find the files using the search.


\begin{figure}[htbp]
\centering
\includegraphics{networkprivacy.png}
\caption{Privacy and distribution of data throughout the network: Only
trusted nodes (blue dots) can access the data from the source (green
dots), while nodes ``friend-of-friend'' (red dots) can't see the data}
\end{figure}


\hypertarget{a-using-multiples-accounts-for-a-group}{%
\subsubsection{a) Using Multiples Accounts For a
Group}\label{a-using-multiples-accounts-for-a-group}}

LAIA allows to use an existing PGP key for signing multiple identities,
and thus makes a team work easy. As detailed before, the PGP keys will be
used for encrypting the SSL passphrase on disk and to sign the
location's SSL certificate.

\hypertarget{b-files-hashing}{%
\subsubsection{b) Files Hashing}\label{b-files-hashing}}

LAIA stores files versioning them through the 40-character
secure hash algorithm SHA-1. In 2017, some concerns came out since
researchers had announced a hash collision of the SHA-1. A hash
collision occurs when two separate inputs produce the same hash result
output.

Although a hash collision is unlikely to occur, there is an important
difference between using cryptographic hash for secure signing, and
using it for generating a file content identifier. In the case of a hash
used for security it is used to trusting users in order to protect people
using a given platform for malicious users. On the other hand, in
content-addressable systems, like in LAIA, the hash for content
identifier is not used to trusting users but, as previously said, for file
content identification.

\hypertarget{c-estabilishing-a-secure-connection}{%
\subsubsection{c) Estabilishing a secure
connection}\label{c-estabilishing-a-secure-connection}}

LAIA's privacy is protected with anonymous tunnels. The swarms often
consist of computers distributed around the world (in which national
laws cannot actually achieve the censorship), and its structure is
independent from a centralized server controlled by corporations or
nation-states. Thus, there is no single entity to sue or pressure
financially, and the data sharing overcomes the infrastructure
jurisdiction and implication in giving users' data to intelligence
agencies.

The secure connection of two nodes is done using the PGP signature that
authenticates SSL links between them and only the trusted nodes can
access details like each other's IP addresses, which files are being
shared, etc. One way to improve the privacy and anonymity is using
the Retro-Tor tool that make it possible to run LAIA over the Tor
network, where IP is not visible even to connected nodes. An advantage
for users is that it creates a hidden node with one click away that
enhances the secure data sharing, and overcomes digital embargo or
censorship. This is an important tool for activities that require a
secure and private contact with a trusted persons.

\hypertarget{conclusion}{%
\section{Conclusion}\label{conclusion}}

The LAIA database proposes a new, efficient, strong and secure
alternative to the centralized structure of environmental data
storage with many advantages comparing to both centralized and
decentralized platforms. The motivations of the project, based on
autonomy, social participation, and freedom of speech are aligned
with the essential principles for the improvement of researches
and the social role of science.

The reputation system, as already pointed out by other projects, is an
effective system of self-management, since the majority of the nodes in
the networks are interested to maintain the system stable and free from
trolls and distractions. Furthermore, it is faster, safer and ensures
that the data will never disappear as there is at least one node hosting
it. The data are private until a node allows a trusted one to access to
its data. It is also easier to overcome the problem with human friendly
names for files and enhances privacy using the Tor network.

The F2F network used by the project is a very strong concept which
allows private sharing of data, and protects other users hosting
data or accessing it. Thus, in LAIA Database the control of access
to the file is not used to restrict the access of information, but
to ensure that researchers groups, collectives connected to the network,
as well as the knowledge itself are safe. It allows data sharing between
trusted nodes by using a free and open source network with high
scalability gathering universities, citizen scientists and stakeholders
promoting their participation and cooperation in the network.

\newpage

\hypertarget{references}{%
\section*{References}\label{references}}
\addcontentsline{toc}{section}{References}

\hypertarget{refs}{}
\leavevmode\hypertarget{ref-de_pryck_denier_2017}{}%
De Pryck, Kari, and François Gemenne. 2017. ``The Denier-in-Chief:
Climate Change, Science and the Election of Donald J. Trump.'' \emph{Law
and Critique} 28 (2): 119--26.
\url{https://doi.org/10.1007/s10978-017-9207-6}.

\leavevmode\hypertarget{ref-halpin_decentralization_2016}{}%
Halpin, Harry, and Alexandre Monnin. 2016. ``The Decentralization of
Knowledge: How Carnap and Heidegger Influenced the Web.'' \emph{First
Monday} 21 (12). \url{https://doi.org/10.5210/fm.v21i12.7109}.

\leavevmode\hypertarget{ref-sargsyan_data_2016}{}%
Sargsyan, Tatevik. 2016. ``Data Localization and the Role of
Infrastructure for Surveillance, Privacy, and Security.''
\emph{International Journal of Communication} 10 (0): 17.
\url{https://ijoc.org/index.php/ijoc/article/view/3854}.

\end{document}


